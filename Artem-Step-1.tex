\documentclass[12pt]{article}
\usepackage[utf8]{inputenc}
\usepackage[T1]{fontenc}
\usepackage{amsmath,amsfonts,amssymb}
\usepackage{graphicx}
\usepackage{a4wide}
\title{Reconstructed abstract of the paper ``Generative or Discriminative? Getting the Best of Both Worlds''}
%\author{not specified, not necessary here}
\date{}
\begin{document}
\maketitle
\begin{abstract}
    This paper \cite{Bishop2007} investigates the integration of two complementary machine learning approaches: discriminative and generative models. The focus is on tasks that can benefit from both labeled and unlabeled data, within the framework of semi-supervised learning. The authors propose a novel interpretation that treats discriminatively trained generative models as distinct entities, rather than hybrid combinations of the two approaches. A new framework is introduced, which unifies these methods to leverage their respective strengths. This framework, grounded in Bayesian inference, enables a continuous transition between purely generative and purely discriminative models by adjusting the prior distribution over model parameters. Experimental results, using both synthetic data and object recognition tasks, demonstrate that optimal performance is achieved through a balanced approach, effectively combining the advantages of both methodologies.
\end{abstract}
\paragraph{Keywords:} Generative models, Discriminative models, Semi-supervised learning, Model interpolation

\paragraph{Highlights:}
\begin{enumerate}
\item Combination of Generative and Discriminative Models.
\item Distinct Model Interpretation for full generative and generative with discriminative learning.
\item automatic determination of the best trade-off between generative and discriminative modeling
\end{enumerate}

\bibliographystyle{unsrt}
\bibliography{Artem-theArt}
\end{document}
