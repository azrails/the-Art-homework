\documentclass[12pt]{article}
\usepackage[utf8]{inputenc}
\usepackage[T1]{fontenc}
\usepackage{amsmath,amsfonts,amssymb}
\usepackage{graphicx}
\usepackage{a4wide}\title{Customer demand forecasting}
%\author{not specified}
\date{}
\begin{document}
\maketitle

%\begin{abstract}
Role: Analyst.
%\end{abstract}
% \paragraph{Keywords:} The Art On Scientific Research, Abstract Reconstruction, Please Put Yours 


\section{Planning the industrial research project}
Before planning the research, the analyst and (\textbf{expert}) discuss the key issues. After the long dash~--- our remarks.

\begin{enumerate}
\item The goal of the project is to understand customer behavior on a marketplace and identify which product attributes make products most attractive to consumers. The objective is to predict customer demand not only based on historical data but also on human behavioral patterns. By understanding what drives customer decisions—such as product features, pricing, reviews, or promotional strategies.

The expected outcome is a behaviorally informed demand forecasting model that accounts for both product characteristics and the psychological factors influencing consumer choices.

\item The project aims to solve the applied problem of predicting customer demand by understanding which product attributes and human behavioral factors drive sales on a marketplace. It will be used for reducing costs on unnecessary items. 
To illustrate the results, especially in the context of reducing costs on unnecessary items, here are a few effective ways to present the outcomes visually:
	\begin{enumerate}
	\item Demand Prediction vs. Actual Sales.
	\item Show which products are deemed unnecessary and can be phased out.
	\end{enumerate}

\item
\begin{enumerate} 
	\item Product Attributes Data: Descriptive features of each product such as price, category, brand, customer ratings. 
	\item Customer Behavioral Data: Actions taken by customers (e.g., clicks, add-to-cart, purchases) and the timing of these actions.
	\item Visual attributes of the product.
\end{enumerate}

\item Huber loss
\item The dataset should cover all relevant periods to provide sufficient historical context for forecasting. Gaps in data can lead to inaccurate predictions.
Include records for various product categories, brands, and customer segments to ensure comprehensive representation of the marketplace.
\item For the combination of data introduced above, it is necessary to use a combination of several models, for example, a random forest for tabular properties of products, a convolutional network for extracting patterns from the visual representation of a product.
\end{enumerate}

\section{Research or development?}
The project involves research activities as the model must take a step towards understanding human psychology and the influence of various factors influencing human behavior, thereby expanding the knowledge of computer models about the real world.

\end{document}